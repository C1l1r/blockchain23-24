%!TEX root = ../thesis.tex
% створюємо вступ

\section{Мета практикуму}

Дослідити основні безпекові вимоги OWASP до web-додатків та можливі вразливості децентралізованих додатків. Запропонувати аналогічний список безпекових вимог для деценталізованих додатків.

\subsection{Постановка задачі та варіант}
\begin{tabularx}{\textwidth}{X|X}
	\textbf{Треба виконати} & \textbf{Зроблено} \\
	Дослідити основні вразливості у OWASP & \checkmark \\
	Дослідити можливі вразливості у децентралізованих додатках & \checkmark \\
    Запропонувати список вимог безпеки для децентралізованих додатків & \checkmark \\
\end{tabularx}

\section{Хід роботи/Опис труднощів}
    На початку роботи над практикумом вибрали гуртом 1 варіант. Згідно вибраного варіанту у даній роботі буде розглянуто можливі небезпеки, що можуть спіткати під час створення децентралізованих додатків та створено попередній список по якому можна буде перевірити безпечність децентралізованого додатку. Під час виконання звіту виникала лише одна часова складність.